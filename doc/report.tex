\documentclass[UTF8]{ctexart}

\usepackage{lmodern}
\usepackage[Symbol]{upgreek}

\usepackage{blindtext} % Package to generate dummy text throughout this template

\usepackage[sc]{mathpazo} % Use the Palatino font
\usepackage[T1]{fontenc} % Use 8-bit encoding that has 256 glyphs
\linespread{1.5} % Line spacing - Palatino needs more space between lines
\usepackage{microtype} % Slightly tweak font spacing for aesthetics

\usepackage[hmarginratio=1:1,top=32mm,columnsep=20pt]{geometry} % Document margins
\usepackage[hang, small,labelfont=bf,up,textfont=it,up]{caption} % Custom captions under/above floats in tables or figures
\usepackage{booktabs} % Horizontal rules in tables

\usepackage{lettrine} % The lettrine is the first enlarged letter at the beginning of the text

\usepackage{enumitem} % Customized lists
\setlist[itemize]{noitemsep} % Make itemize lists more compact

\usepackage{abstract} % Allows abstract customization
\renewcommand{\abstractnamefont}{\normalfont\bfseries} % Set the "Abstract" text to bold
\renewcommand{\abstracttextfont}{\normalfont\small, \itshape} % Set the abstract itself to small italic text

\usepackage{fancyhdr} % Headers and footers
\pagestyle{fancy} % All pages have headers and footers
\fancyhead{} % Blank out the default header
\fancyfoot{} % Blank out the default footer

\usepackage{indentfirst}
\setlength{\parindent}{2em}
\usepackage{amsmath}
\usepackage{adjustbox}
\usepackage{graphicx}
\usepackage{longtable}
\usepackage{bm}
\usepackage{floatrow}
\usepackage{enumerate}

\usepackage{titlesec} % Allows customization of titles
\usepackage{titling} % Customizing the title section
\usepackage{minted}

\usepackage{float}
\usepackage{multirow}
\usepackage[table,xcdraw]{xcolor}

%----------------------------------------------------------------------------------------
%	TITLE SECTION
%----------------------------------------------------------------------------------------


\setlength{\droptitle}{-1\baselineskip}
 % Article title closing formatting
\title{\zihao{1}{2017年夏季Java小学期大作业\\ 实验报告}}
\author{陈经基\\2015011358 \and 高信龙一\\2015080060 \and 张蔚\\2015011352}
\date{\today} % Leave empty to omit a date

%----------------------------------------------------------------------------------------

\begin{document}

% Print the title

\maketitle
\hrule

\tableofcontents

\section{功能实现}

\begin{table}[H]
\centering
\begin{tabular}{|l|l|l|c|c|}
\hline
                                                                         & 功能                                                                   & 子功能                                                                                                    & 百分比   & {\color[HTML]{000000} \begin{tabular}[c]{@{}c@{}}是否\\ 实现\end{tabular}} \\ \hline
                                                                         & 系统支持                                                                 & \begin{tabular}[c]{@{}l@{}}要保证程序在安卓机上正常运行,测试过程\\ 中程序不崩溃。\end{tabular}                                  & 5     & {\color[HTML]{FE0000} 是}                                               \\ \cline{2-5} 
                                                                         & 页面布局                                                                 & 布局合理,点击处理正确                                                                                            & 10    & {\color[HTML]{FE0000} 是}                                               \\ \cline{2-5} 
                                                                         & 分类列表                                                                 & 删除和添加操作                                                                                                & 10    & {\color[HTML]{FE0000} 是}                                               \\ \cline{2-5} 
                                                                         &                                                                      & \begin{tabular}[c]{@{}l@{}}正确显示新闻列表的消息,布局和展示,点\\ 击进入新闻详情页面正确。\end{tabular}                             & 10    & {\color[HTML]{FE0000} 是}                                               \\ \cline{3-5} 
                                                                         &                                                                      & \begin{tabular}[c]{@{}l@{}}实现新闻的本地存储,看过的新闻列表在离\\ 线的情况下也可以浏览\end{tabular}                               & 10    & {\color[HTML]{FE0000} 是}                                               \\ \cline{3-5} 
                                                                         &                                                                      & 上拉获取更多新闻                                                                                               & 5     & {\color[HTML]{FE0000} 是}                                               \\ \cline{3-5} 
                                                                         &                                                                      & 新闻是否看过的页面灰色标记                                                                                          & 5     & {\color[HTML]{FE0000} 是}                                               \\ \cline{3-5} 
                                                                         & \multirow{-5}{*}{新闻列表}                                               & 新闻搜索                                                                                                   & 5     & {\color[HTML]{FE0000} 是}                                               \\ \cline{2-5} 
                                                                         &                                                                      & \begin{tabular}[c]{@{}l@{}}使用微信、微博等 SDK分享,新闻详情页面\\ 点击分享可以分享到常用的 app,分享内容\\ 带有新闻摘要、URL 和图片\end{tabular} & 10    & {\color[HTML]{FE0000} 是}                                               \\ \cline{3-5} 
\multirow{-10}{*}{\begin{tabular}[c]{@{}l@{}}基\\ 础\\ 功\\ 能\end{tabular}} & \multirow{-2}{*}{分享收藏}                                               & \begin{tabular}[c]{@{}l@{}}新闻详情页面点击收藏的添加和删除,实现\\ 收藏新闻的本地存储。收藏也的正确展示,\\ 点击可以进入新闻详情等\end{tabular}        & 10    & {\color[HTML]{FE0000} 是}                                               \\ \hline
                                                                         & 新闻推荐                                                                 & \begin{tabular}[c]{@{}l@{}}根据用户看过的新闻推荐相关的新闻,参考\\ 今日头条等\end{tabular}                                    & 10    & {\color[HTML]{FE0000} 是}                                               \\ \cline{2-5} 
                                                                         & 语音播报                                                                 & 可以语音读出新闻等                                                                                              & 20    & {\color[HTML]{FE0000} 是}                                               \\ \cline{2-5} 
                                                                         & \begin{tabular}[c]{@{}l@{}}新闻人物\\ 地点链接\end{tabular}                  & 用户可以跳转到相应的百科词条等                                                                                        & 5     & {\color[HTML]{FE0000} 是}                                               \\ \cline{2-5} 
                                                                         &                                                                      & \begin{tabular}[c]{@{}l@{}}新闻屏蔽功能,通过进一步询问用户想要屏\\ 蔽掉关于什么内容的新闻而实现基于关键词\\ 的屏蔽等\end{tabular}               & 5     & {\color[HTML]{FE0000} 是}                                               \\ \cline{3-5} 
                                                                         &                                                                      & 夜间模式,用户可以调整背景色等                                                                                        & 5     & {\color[HTML]{FE0000} 是}                                               \\ \cline{3-5} 
                                                                         &                                                                      & \begin{tabular}[c]{@{}l@{}}文字模式和图片模式转换,文字模式不显示\\ 图片,帮助用户节省流量。\end{tabular}                             & 5     & {\color[HTML]{FE0000} 是}                                               \\ \cline{3-5} 
                                                                         &                                                                      & 流畅性强                                                                                                   & 视效果而定 & {\color[HTML]{FE0000} }                                                \\ \cline{3-5} 
                                                                         &                                                                      & 界面颜值高                                                                                                  & 视效果而定 & {\color[HTML]{FE0000} }                                                \\ \cline{3-5} 
                                                                         & \multirow{-6}{*}{用户体验}                                               & \begin{tabular}[c]{@{}l@{}}根据新闻文本补上相关的图片(因为部分新\\ 闻内容没有图片)\end{tabular}                                 & 视效果而定 & {\color[HTML]{FE0000} }                                                \\ \cline{2-5} 
                                                                         &                                                                      & 使用了较好的框架                                                                                               & 酌情加分  & {\color[HTML]{FE0000} }                                                \\ \cline{3-5} 
                                                                         &                                                                      & 有较完整的单元测试                                                                                              & 5     & {\color[HTML]{FE0000} 是}                                               \\ \cline{3-5} 
                                                                         &                                                                      & 使用了 github 等好用的代码版本管理工具                                                                                & 2     & {\color[HTML]{FE0000} 是}                                               \\ \cline{3-5} 
                                                                         &                                                                      & maven 或者 gradle 等项目管理工具                                                                                & 2     & {\color[HTML]{FE0000} 是}                                               \\ \cline{3-5} 
\multirow{-14}{*}{\begin{tabular}[c]{@{}l@{}}加\\ 分\\ 功\\ 能\end{tabular}} & \multirow{-5}{*}{\begin{tabular}[c]{@{}l@{}}代码和\\ 项目管理\end{tabular}} & 其他提高编码效率的工具或操作                                                                                         & 酌情加分  & {\color[HTML]{FE0000} }                                                \\ \hline
\end{tabular}
\end{table}

\section{小组分工}

\begin{table}[H]
\centering
\begin{tabular}{|l|l|c|c|}
\hline
功能                                                                   & 子功能                                                                                                    & 百分比   & {\color[HTML]{000000} 实现人} \\ \hline
系统支持                                                                 & \begin{tabular}[c]{@{}l@{}}要保证程序在安卓机上正常运行,测试过程\\ 中程序不崩溃。\end{tabular}                                  & 5     & {\color[HTML]{FE0000} 张三}  \\ \hline
页面布局                                                                 & 布局合理,点击处理正确                                                                                            & 10    & {\color[HTML]{FE0000} 李四}  \\ \hline
分类列表                                                                 & 删除和添加操作                                                                                                & 10    & {\color[HTML]{FE0000} 王五}  \\ \hline
                                                                     & \begin{tabular}[c]{@{}l@{}}正确显示新闻列表的消息,布局和展示,点\\ 击进入新闻详情页面正确。\end{tabular}                             & 10    & {\color[HTML]{FE0000} 是}   \\ \cline{2-4} 
                                                                     & \begin{tabular}[c]{@{}l@{}}实现新闻的本地存储,看过的新闻列表在离\\ 线的情况下也可以浏览\end{tabular}                               & 10    & {\color[HTML]{FE0000} 是}   \\ \cline{2-4} 
                                                                     & 上拉获取更多新闻                                                                                               & 5     & {\color[HTML]{FE0000} 是}   \\ \cline{2-4} 
                                                                     & 新闻是否看过的页面灰色标记                                                                                          & 5     & {\color[HTML]{FE0000} 是}   \\ \cline{2-4} 
\multirow{-5}{*}{新闻列表}                                               & 新闻搜索                                                                                                   & 5     & {\color[HTML]{FE0000} 是}   \\ \hline
                                                                     & \begin{tabular}[c]{@{}l@{}}使用微信、微博等 SDK分享,新闻详情页面\\ 点击分享可以分享到常用的 app,分享内容\\ 带有新闻摘要、URL 和图片\end{tabular} & 10    & {\color[HTML]{FE0000} 是}   \\ \cline{2-4} 
\multirow{-2}{*}{分享收藏}                                               & \begin{tabular}[c]{@{}l@{}}新闻详情页面点击收藏的添加和删除,实现\\ 收藏新闻的本地存储。收藏也的正确展示,\\ 点击可以进入新闻详情等\end{tabular}        & 10    & {\color[HTML]{FE0000} 是}   \\ \hline
新闻推荐                                                                 & \begin{tabular}[c]{@{}l@{}}根据用户看过的新闻推荐相关的新闻,参考\\ 今日头条等\end{tabular}                                    & 10    & {\color[HTML]{FE0000} 是}   \\ \hline
语音播报                                                                 & 可以语音读出新闻等                                                                                              & 20    & {\color[HTML]{FE0000} 是}   \\ \hline
\begin{tabular}[c]{@{}l@{}}新闻人物\\ 地点链接\end{tabular}                  & 用户可以跳转到相应的百科词条等                                                                                        & 5     & {\color[HTML]{FE0000} 是}   \\ \hline
                                                                     & \begin{tabular}[c]{@{}l@{}}新闻屏蔽功能,通过进一步询问用户想要屏\\ 蔽掉关于什么内容的新闻而实现基于关键词\\ 的屏蔽等\end{tabular}               & 5     & {\color[HTML]{FE0000} 是}   \\ \cline{2-4} 
                                                                     & 夜间模式,用户可以调整背景色等                                                                                        & 5     & {\color[HTML]{FE0000} 是}   \\ \cline{2-4} 
                                                                     & \begin{tabular}[c]{@{}l@{}}文字模式和图片模式转换,文字模式不显示\\ 图片,帮助用户节省流量。\end{tabular}                             & 5     & {\color[HTML]{FE0000} 是}   \\ \cline{2-4} 
                                                                     & 流畅性强                                                                                                   & 视效果而定 & {\color[HTML]{FE0000} }    \\ \cline{2-4} 
                                                                     & 界面颜值高                                                                                                  & 视效果而定 & {\color[HTML]{FE0000} }    \\ \cline{2-4} 
\multirow{-6}{*}{用户体验}                                               & \begin{tabular}[c]{@{}l@{}}根据新闻文本补上相关的图片(因为部分新\\ 闻内容没有图片)\end{tabular}                                 & 视效果而定 & {\color[HTML]{FE0000} }    \\ \hline
                                                                     & 使用了较好的框架                                                                                               & 酌情加分  & {\color[HTML]{FE0000} }    \\ \cline{2-4} 
                                                                     & 有较完整的单元测试                                                                                              & 5     & {\color[HTML]{FE0000} 是}   \\ \cline{2-4} 
                                                                     & 使用了 github 等好用的代码版本管理工具                                                                                & 2     & {\color[HTML]{FE0000} 是}   \\ \cline{2-4} 
                                                                     & maven 或者 gradle 等项目管理工具                                                                                & 2     & {\color[HTML]{FE0000} 是}   \\ \cline{2-4} 
\multirow{-5}{*}{\begin{tabular}[c]{@{}l@{}}代码和\\ 项目管理\end{tabular}} & 其他提高编码效率的工具或操作                                                                                         & 酌情加分  & {\color[HTML]{FE0000} }    \\ \hline
\end{tabular}
\end{table}

\section{具体实现}

\section{总结与心得}

%----------------------------------------------------------------------------------------

\end{document}

